%%%%%%%%%%%%%%%%%%%%%%%%%%%%%%%%%%%%%%%%%%%%%%%%%%%%%%%%%%%%%%%%
% Preamble
\documentclass[12pt]{article}

%%%%%%%%%%%%%%%%%%%%%%%%%%%%%%%%%%%%%%%%%%%%%%%%%%%%%%%%%%%%%%%%
% Input Commands and Packages
%%%%%%%%%%%%%%%%%%%%%%%%%%%%%%%%%%%%%%%%
% Packages
%\usepackage[pdftex]{color}
%\usepackage[pdftex]{graphicx}
%\usepackage[pdftex,colorlinks=true,linkcolor=black,pdfview=FitH,pdfstartview=FitH]{hyperref}
%\hypersetup{citecolor=black}
%\pdfcompresslevel 9

\usepackage{amsfonts}
\usepackage{amsmath}
\usepackage{bm}                 % bold math

%%%%%%%%%%%%%%%%%%%%%%%%%%%%%%%%%%%%%%%%
% Labels and References

\newcommand{\alabel}[1]{\label{app:#1}}
\newcommand{\clabel}[1]{\label{cpt:#1}}
\newcommand{\sslabel}[1]{\label{sec:#1}}
\newcommand{\elabel}[1]{\label{eqn:#1}}
\newcommand{\flabel}[1]{\label{fig:#1}}
\newcommand{\tlabel}[1]{\label{tab:#1}}

\newcommand{\aref}[1]{\ref{app:#1}}
\newcommand{\cref}[1]{\ref{cpt:#1}}
\newcommand{\sref}[1]{\ref{sec:#1}}
\newcommand{\eref}[1]{\ref{eqn:#1}}
\newcommand{\fref}[1]{\ref{fig:#1}}
\newcommand{\tref}[1]{\ref{tab:#1}}

% sections with labels
\newcommand{\MEsec}[2]{\section{\sslabel{#2}#1}}
\newcommand{\MEssec}[2]{\subsection{\sslabel{#2}#1}}
\newcommand{\MEsssec}[2]{\subsubsection{\sslabel{#2}#1}}

%%%%%%%%%%%%%%%%%%%%%%%%%%%%%%%%%%%%%%%%
% Structures

% List
\newcommand{\listBegin}{\begin{tabular}{cp{4.5in}}}
\newcommand{\listEnd}{\end{tabular}}
\newcommand{\listItem}{\ensuremath{\bullet}&}

% Matrix
\newcommand{\matrixBegin}[1]{\left[\!\!\left[ \begin{array}{#1}}
\newcommand{\matrixEnd}{\end{array} \right]\!\!\right]}

%%%%%%%%%%%%%%%%%%%%%%%%%%%%%%%%%%%%%%%%
% Environments

% equations
\newcommand{\beq}[1]{\begin{equation}\elabel{#1}}
\newcommand{\eeq}{\end{equation}}

\newcommand{\beqa}[1]{\begin{eqnarray}\elabel{#1}}
\newcommand{\eeqa}{\end{eqnarray}}

% figures
%  MEfig{figure args}{width}{fig name}{caption}
%  MEfig{h}{0.9}{optFP}{A Fabry-Perot cavity.}
\newcommand{\MEfig}[4]{\begin{figure}[#1] \begin{center}\includegraphics[width=#2\linewidth]{#3}
 \vspace{-5pt} \caption{\flabel{#3}#4}\end{center}\end{figure}}

%%%%%%%%%%%%%%%%%%%%%%%%%%%%%%%%%%%%%%%%
% Misc
\newcommand{\vecv}[1]{\vec{v}_{#1}}
\newcommand{\matm}[1]{\mathbf{M}_{#1}}
\newcommand{\matI}{\mathbb{I}}

% left-right groupings
\newcommand{\abs}[1]{\left| #1 \right|}
\newcommand{\grp}[1]{\left[ #1 \right]}
\newcommand{\prn}[1]{\left( #1 \right)}
\newcommand{\set}[1]{\left\{ #1 \right\}}

\def\({\left(}
\def\){\right)}
\def\[{\left[}
\def\]{\right]}

% text in equations
\def\where{\text{where}}
\def\suchthat{\text{such that}}
\def\with{\text{with}}
\def\and{\text{and}}
\def\for{\text{for}}
\def\since{\text{since}}
\newcommand{\undernote}[2]{\underbrace{#1}_{\text{#2}}}

% functions
\newcommand{\fun}[2]{\,{#1}\!\left( {#2} \right)}
\newcommand{\funt}[1]{\fun{#1}{t}}
\newcommand{\tfun}[2]{\fun{\textrm{#1}}{#2}}
\newcommand{\tfunt}[1]{\funt{\textrm{#1}}}

\newcommand{\fsin}[1]{\sin\!\left( {#1} \right)}
\newcommand{\fcos}[1]{\cos\!\left( {#1} \right)}
\newcommand{\fexp}[1]{\mathcal{e}^{#1}}

\newcommand{\imag}[1]{Im\!\left[ {#1} \right]}
\newcommand{\real}[1]{Re\!\left[ {#1} \right]}

% partial derivatives
\newcommand{\deriv}[2]{\frac{d{#1}}{d{#2}}}
\newcommand{\partderiv}[2]{\frac{\partial{#1}}{\partial{#2}}}
\newcommand{\partderivT}[2]{\partial{#1}/\partial{#2}}
\newcommand{\partderivS}[2]{\frac{\partial^2{#1}}{\partial{#2}^2}}

\newcommand{\dint}[3]{\int_{#1}^{#2} \hspace{-2ex} #3 ~}


% half and quarter
\newcommand{\half}{\tfrac{1}{2}}
\newcommand{\qtr}{\tfrac{1}{4}}

% gravitational waves
\def\gw{gravitational wave}
\def\gws{gravitational waves}
\def\GW{Gravitational Wave}
\def\GWs{Gravitational Waves}

% common symbols
\def\kB{k_B}
\def\fins{\mathcal{F}}

% ordinal numbers
\def\first{1\textsuperscript{st}}
\def\second{2\textsuperscript{nd}}
\def\third{3\textsuperscript{rd}}
\def\forth{4\textsuperscript{th}}

%%%%%%%%%%%%%%%%%%%%%%%%%%%%%%%%%%%%%%%%
% Units
\usepackage{siunitx}

\sisetup{per-mode = symbol-or-fraction}
\sisetup{load-configurations = abbreviations}

\DeclareSIUnit\rtHz{$\sqrt{\si{\Hz}}$}
\DeclareSIUnit\mrtHz{\meter\per\rtHz}

%%%%%%%%%%%%%%%%%%%%%%%%%%%%%%%%%%%%%%%%
% Colored Text
\definecolor{spring}{rgb}{0.7,0.9,0.7}
\definecolor{brick}{rgb}{0.7,0.2,0.1}
\definecolor{redHL}{rgb}{1.0,0.5,0.5}
\newcommand{\markit}[1]{\colorbox{spring}{#1}}
\newcommand{\fixme}[1]{\colorbox{redHL}{#1}}


%%%%%%%%%%%%%%%%%%%%%%%%%%%%%%%%%%%%%%%%%%%%%%%%%%%%%%%%%%%%%%%%
% Document Specific Abbreviations

% prose abbreviations
\def\DOF{Degree of Freedom}
\def\DSF{Degrees of Freedom}
\def\dof{degree of freedom}
\def\dsf{degrees of freedom}

\def\Ifo{Interferometer}
\def\ifo{interferometer}

% common numbers, vectors and matrices
\def\Nlnk{N_{link}}
\def\Nrf{N_{RF}}
\def\Nfld{N_{field}}
\def\Nopt{N_{optic}}
\def\Nprb{N_{probe}}

\def\vFrf{\vecv{f_{RF}}}
\def\vDC{\vecv{DC}}
\def\vSrc{\vecv{source}}
\def\vLen{\vecv{length}}
\def\vAC{\vecv{AC}}
\def\vSig{\vecv{signal}}
\def\vExc{\vecv{excitation}}

\def\mDC{\matm{DC}}
\def\mAC{\matm{AC}}
\def\mTF{\matm{TF}}
\def\mOpt{\matm{opt}}
\def\mOut{\matm{out}}
\def\mIn{\matm{in}}
\def\mPhi{\matm{\phi}}
\def\mPrb{\matm{prb}}
\def\mDOF{\matm{DOF}}
\def\mRct{\matm{rct}}
\def\mGen{\matm{gen}}
\def\mDrv{\matm{mod}}

\newcommand{\mOptn}[1]{\matm{opt_{#1}}}
\newcommand{\mOutn}[1]{\matm{out_{#1}}}
\newcommand{\mInn}[1]{\matm{in_{#1}}}
\newcommand{\mPhin}[1]{\matm{\phi_{#1}}}
\newcommand{\mPrbn}[1]{\matm{prb_{#1}}}
\newcommand{\mPrbInn}[1]{\matm{pin_{#1}}}
\newcommand{\mDOFn}[1]{\matm{DOF_{#1}}}
\newcommand{\mRctn}[1]{\matm{rct_{#1}}}
\newcommand{\mDrvn}[1]{\matm{drv_{#1}}}


%%%%%%%%%%%%%%%%%%%%%%%%%%%%%%%%%%%%%%%%%%%
% EXAMPLES

% length notation
\newcommand{\Len}[2]{L_{#1:#2}}
\newcommand{\LenFPC}{L}

% field amplitude notation
\newcommand{\FieldFrontOut}[3][]{#2_{#3}^{_{fo}\,#1}}
\newcommand{\FieldFrontIn}[3][]{#2_{#3}^{_{fi}\,#1}}

% power signals
\newcommand{\Pin}{P_{IN}}
\newcommand{\Ptrx}{\ensuremath{P_{TR\{T,R\}}}}
\newcommand{\bPtrx}{\ensuremath{\bar{P}_{TR\{T,R\}}}}

% input matrix and DOFs
\newcommand{\SM}{\mathbf{M}}
\newcommand{\tSM}{\mathbf{\tilde{M}}}
\newcommand{\IM}{\mathbf{G}}

\newcommand{\lp}{PRC}

\newcommand{\dl}{\Delta}
\newcommand{\vdl}{\vec{\dl}}
\newcommand{\dlp}{\dl_{\lp}}

\newcommand{\slp}{S_{err_{\lp}}}

% fields in appendix A
\newcommand{\Ein}{E_{IN}}
\newcommand{\etaMi}{\eta}
\newcommand{\etaM}{{\group{\eta_R + \eta_I}}}
\newcommand{\etaP}{{\group{\eta_R - \eta_I}}}

% misc symbols
\newcommand{\fin}{\mathcal{F}}
\newcommand{\Jmod}[1]{\fun{J_{#1}}{\Gamma_{mod}}}
\newcommand{\wn}{k} 

%%%%%%%%%%%%%%%%%%%%%%%%%%%%%%%%%%%%%%%%%%%
% ENVIRONMENTS

\newenvironment{funcinfo}[1]
  {\vspace{0.1in}
    \begin{tabular}{l}
      \footnotesize \it #1\\
      \hline
      \scriptsize
      \begin{tabular}{ll}}
  {   \end{tabular}
     \end{tabular}
   \vspace{0.1in}}

\newenvironment{funcinfo2}[2]
  {\vspace{0.1in}
    \begin{tabular}{l}
      \footnotesize \it #1\\
      \footnotesize \it #2\\
      \hline
      \scriptsize
      \begin{tabular}{ll}}
  {   \end{tabular}
     \end{tabular}
   \vspace{0.1in}}


\newcommand{\tit}[1]{\begin{center}\textbf{\Huge{#1}}\end{center}\vspace{1ex}}
\newcommand{\fig}[1]{\textbf{\large{Figure #1}}}

%%%%%%%%%%%%%%%%%%%%%%%%%%%%%%%%%%%%%%%%%%%%%%%%%%%%%%%%%%%%%%%%
% Header

%%%%%%%%%%%%%%%%%%%%%%%%%%%%%%%%%%%%%%%%%%%%%%%%%%%%%%%%%%%%%%%%
% Document

\begin{document}

%%%%%%%%%%%%%%%% PDH
\tit{Pound-Drever-Hall Error Signal}
\fig{3.2}

\begin{center}
\includegraphics[width=0.9\textwidth]{pdh.eps}
\end{center}

The Pound-Drever-Hall error signal for a \FPC.
 The cavity parameters are similar to those of the LIGO 1 arm
  cavities ($r_{IT} = 0.986$, $r_{ET} = 1$, and $\lambda_0 = 1064\nm$.)
 The ``linear region'' in $S_{PDH}$  is centered at $\Delta = 0$ and
  approximately $1\nm$ wide.
 In this region a standard linear controller can be used to
  hold the cavity on resonance.
 The carrier power in the cavity,
  $\abs{\Afo{IT_0}}^2$, is also shown for reference.

\begin{equation}
S_{PDH} \propto \abs{\Afo{IT_0}}^2 \fsin{2 \wn_0 \Delta}
\end{equation}


\newpage
%%%%%%%%%%%%%%%% TVa
\tit{Threshold Velocity: Naive Model}

\fig{3.3}

\begin{center}
\includegraphics[width=0.9\textwidth]{threshA.eps}
\end{center}

 Threshold velocity in a simple \la{} model.
 A linear controller attempts to lock the cavity as
  $\Delta = 0$ is approached with various initial velocities
  $v_{init}$.
 From left to right, the first is well below the threshold velocity,
  the second just below, and the third well above.

\newpage
%%%%%%%%%%%%%%%% TVb
\tit{Threshold Velocity: Slightly Less Naive Model}

\fig{3.4}

\begin{center}
\includegraphics[width=0.9\textwidth]{threshB.eps}
\end{center}

 Threshold velocity in a more realistic \la{} model.
 Including realistic actuation limits significantly
  reduces the threshold velocity of a linear controller.
 The model parameters are taken from the LIGO 1 arm cavity:
  the force limit is $10 \mN$
  and the mass of the optic is $10.3 \kg$.

\newpage
%%%%%%%%%%%%%%%% ESL
\tit{Error Signal Linearization}

\fig{3.5}

\begin{center}
\includegraphics[width=0.9\textwidth]{Slin.eps}
\end{center}

 $S_{Lin}$ is scaled by $P_{TRN}$ at $\Delta = 0$ such that
  the slopes of $S_{Lin}$ and $S_{PDH}$ are equal near the
  resonance point and
  $\abs{\Afo{IT_0}}^2 = P_{TRN} / t_{ET}^2$ is shown for reference.
 The broad linear region in $S_{Lin}$ makes it a superior
  error signal for use with a linear controller,
  especially during \la{}.

\begin{equation}
S_{PDH} \propto \abs{\Afo{IT_0}}^2 \fsin{2 \wn_0 \Delta}
\end{equation}

\begin{equation}
S_{Lin} = \frac{S_{PDH}}{P_{TRN}}
\end{equation}

\begin{equation}
S_{Lin} \propto \fsin{2 \wn_0 \Delta}
\end{equation}

\begin{equation}
S_{Lin} \sim \Delta
\end{equation}


\newpage
%%%%%%%%%%%%%%%% LA with ESL
\tit{\LA{} with Error Signal Linearization: Simulated}

\fig{3.6}

\begin{center}
\includegraphics[width=0.9\textwidth]{threshC.eps}
\end{center}

 Threshold velocity with error signal linearization, simulated.
 Figure (a) shows $\Delta$ and the force applied to the corresponding
  \dof{} during a simulated \la{} event.
 The power in the cavity ($\abs{\Afo{IT_0}}^2$, dash-dot),
  demod signal ($S_{demod}$, solid),
  and linearized error signal ($S_{Lin}$, dashed)
  are shown in (b) for the same event.
 Note that the threshold velocity of this controller is more than
  10 times greater than that shown in figure 3.4 for
  a controller without error signal linearization,
  despite having identical actuation limitations.

\newpage
%%%%%%%%%%%%%%%% LA with ESL
\tit{\LA{} with Error Signal Linearization: Experimental}

\fig{3.7}

\begin{center}
$v_{init} \sim 1 \um/s$
\end{center}
\begin{center}
\includegraphics[width=0.9\textwidth]{threshD.eps}
\end{center}

 This data was collected at the LIGO Hanford Observatory using one of the
  $2 \km$ arm cavities.
 This event, which has $v_{init} \sim 1 \um/\second$, is very near the
  threshold velocity of the controller.
 Note how $S_{Lin}$ continues to grow even after the
  linear region in $S_{demod}$ has been crossed,
  thereby allowing the controller to acquire lock when it would have
  otherwise been lost.

\newpage
%%%%%%%%%%%%%%%% 1
\tit{States 1 and 2}

\input{state1.pstex_t}

None of the \dsf{} are controlled.
The cavities occasionally resonate as the mirrors move freely.
This is the starting point for lock acquisition.

\begin{equation*}
\matrixBegin{c}
 \slp \\
 \sLp \\
 \sLm \\
 \slm \\
\matrixEnd
 =
\matrixBegin{c}
 0 \\
 0 \\
 0 \\
 0 \\
\matrixEnd
\end{equation*}

\vspace{0.5in}

\input{state2.pstex_t}

$\dlm$ and $\dlp$
 are controlled such that the carrier is anti-resonant in the
 power recycling cavity and zero at the ASY port.
The recycling cavity length and the modulation frequency are
 chosen such that the carrier anti-resonance is
 coincident with a resonance for the first-order sidebands.

The sideband power in the recycling cavity in this state,
 and throughout the rest of the state progression,
 is about 10 times the input sideband power.
There is essentially no carrier power in the \ifo{} in this state.

\begin{equation*}
\matrixBegin{c}
 \slp \\
 \slm \\
\matrixEnd
 =
\matrixBegin{cc}
 G_{\lp, I_{ref}} & 0 \\
 0                & G_{\lm, Q_{ref}} \\
\matrixEnd^{-1}
\matrixBegin{c}
 I_{ref} \\
 Q_{ref} \\
\matrixEnd
\end{equation*}

\begin{equation*}
\matrixBegin{c}
 \sLp \\
 \sLm \\
\matrixEnd
 =
\matrixBegin{c}
 0 \\
 0 \\
\matrixEnd
\end{equation*}

\newpage
%%%%%%%%%%%%%%%% 3
\tit{State 3}

\input{state3.pstex_t}

State 3 is reached when state 2 holds and one of the two arm cavities
 is controlled such that the carrier is resonant.
Resonance in the arm cavity causes the carrier field
 reflected from that arm to reverse its sign,
 thereby making the ASY port bright for the carrier.

\begin{equation*}
\matrixBegin{c}
 \slp \\
 \sLp \\
 \slm \\
\matrixEnd
 =
\matrixBegin{ccc}
 G_{\lp, I_{ref}} & G_{\Lp, I_{ref}} & 0 \\
 0                & G_{\Lp, Q_{asy}} & 0 \\
 0                & 0 & G_{\lm, Q_{ref}} \\
\matrixEnd^{-1}
\matrixBegin{c}
 I_{ref} \\
 Q_{asy} \\
 Q_{ref} \\
\matrixEnd
\end{equation*}

In this state $\sLm = \pm \sLp$ depending on which arm is locked.

\newpage
%%%%%%%%%%%%%%%% 4
\tit{State 4}

\input{state4.pstex_t}

This is a transitory state that occurs when state 3 holds and
 the as yet uncontrolled arm cavity is locked at carrier resonance.
In this state the carrier is resonant in both arm cavities and the
 recycling cavity.
The resulting coupled
 cavity allows the carrier power in the \ifo{} to increase
 by roughly three orders of magnitude.

At the onset of this state all of the \dsf{} are controlled and all of the
 ``No Signal'' singularities have been removed from $\SM$.
\begin{equation*}
\matrixBegin{c}
 \slp \\
 \sLp \\
 \sLm \\
 \slm \\
\matrixEnd
 =
\matrixBegin{cccc}
 G_{\lp, I_{ref}} & G_{\Lp, I_{ref}} & G_{\Lm, I_{ref}} & 0 \\
 G_{\lp, I_{pob}} & G_{\Lp, I_{pob}} & G_{\Lm, I_{pob}} & 0 \\
 0                & G_{\Lp, Q_{asy}} & G_{\Lm, Q_{asy}} & 0 \\
 0                & 0 & 0 & G_{\lm, Q_{ref}} \\
\matrixEnd^{-1}
\matrixBegin{c}
 I_{ref} \\
 I_{pob} \\
 Q_{asy} \\
 Q_{ref} \\
\matrixEnd
\end{equation*}

As the carrier power in the recycling cavity builds,
 and the reflected carrier power decreases,
 control of $\dlm$ is switched to $Q_{pob}$ to avoid a
 zero in the $Q_{ref}$ signal.
\begin{equation*}
\matrixBegin{c}
 \slp \\
 \sLp \\
 \sLm \\
 \slm \\
\matrixEnd
 =
\matrixBegin{cccc}
 G_{\lp, I_{ref}} & G_{\Lp, I_{ref}} & G_{\Lm, I_{ref}} & 0 \\
 G_{\lp, I_{pob}} & G_{\Lp, I_{pob}} & G_{\Lm, I_{pob}} & 0 \\
 0                & G_{\Lp, Q_{asy}} & G_{\Lm, Q_{asy}} & 0 \\
 0                & 0 & 0 & G_{\lm, Q_{pob}} \\
\matrixEnd^{-1}
\matrixBegin{c}
 I_{ref} \\
 I_{pob} \\
 Q_{asy} \\
 Q_{pob} \\
\matrixEnd
\end{equation*}

A ``Degenerate Signal'' singularity is encountered in the course of
 the power buildup.
As the singularity is approached,
 control of $\dlp$ is relinquished,
 but is regained once the singularity has
 passed enroute to state 5.
\begin{equation*}
\matrixBegin{c}
 \sLp \\
 \sLm \\
 \slm \\
\matrixEnd
 =
\matrixBegin{ccc}
 G_{\Lp, I_{ref}} & G_{\Lm, I_{ref}} & 0 \\
 G_{\Lp, Q_{asy}} & G_{\Lm, Q_{asy}} & 0 \\
 0 & 0 & G_{\lm, Q_{ref}} \\
\matrixEnd^{-1}
\matrixBegin{c}
 I_{ref} \\
 Q_{asy} \\
 Q_{ref} \\
\matrixEnd
\end{equation*}

\begin{equation*}
 \slp = 0
\end{equation*}

\newpage
%%%%%%%%%%%%%%%% 5
\tit{State 5}

\input{state5.pstex_t}

The final state of the \ifo, at least from the \la{} point of view,
 is reached when state 4 has endured long enough for the power in
 the \ifo{} to stabilize.
This is the ending point for lock acquisition, though
 the controllers used to achieve this state must be capable of
 holding it long enough for the transition to a low-noise controller
 to occur.

\begin{equation*}
\matrixBegin{c}
 \slp \\
 \sLp \\
 \sLm \\
 \slm \\
\matrixEnd
 =
\matrixBegin{cccc}
 G_{\lp, I_{ref}} & G_{\Lp, I_{ref}} & G_{\Lm, I_{ref}} & 0 \\
 G_{\lp, I_{pob}} & G_{\Lp, I_{pob}} & G_{\Lm, I_{pob}} & 0 \\
 0                & G_{\Lp, Q_{asy}} & G_{\Lm, Q_{asy}} & 0 \\
 0                & 0 & 0 & G_{\lm, Q_{pob}} \\
\matrixEnd^{-1}
\matrixBegin{c}
 I_{ref} \\
 I_{pob} \\
 Q_{asy} \\
 Q_{pob} \\
\matrixEnd
\end{equation*}

\newpage
%%%%%%%%%%%%%%%% LA
\tit{\LA}

\fig{4.2}

\begin{center}
\includegraphics[width=0.9\textwidth]{lockA.eps}
\end{center}

\vspace{0.5in}

\fig{5.5}

\begin{center}
\includegraphics[width=0.9\textwidth]{lockB.eps}
\end{center}

\newpage
%%%%%%%%%%%%%%%% LA
\tit{Gain Coefficients}

\begin{eqnarray}
 G_{\Lp, I_{ref}} & = & g_{Aref}~ \Any{REF_1}~ A_{+}  \\
 G_{\Lm, I_{ref}} & = & g_{Aref}~ \Any{REF_1}~ A_{-} \\
 G_{\lp, I_{ref}}  & = & g_{Pref}~ \group{\Any{REF_0} - \Any{REF_2}}~
  A_{PRM} \\
 G_{\Lp, I_{pob}} & = & g_{Apob}~ \Any{POB_1}~ A_{+} \\
 G_{\Lm, I_{pob}} & = & g_{Apob}~ \Any{POB_1}~ A_{-} \\
 G_{\lp, I_{pob}}  & = & g_{Ppob}~ \Any{POB_0}~ A_{PRM} \\
 G_{\Lp, Q_{asy}} & = & g_{Aasy}~ \Any{ASY_1}~ A_{-} \\
 G_{\Lm, Q_{asy}} & = & g_{Aasy}~ \Any{ASY_1}~ A_{+} \\
 G_{\lm, Q_{ref}} & = & g_{Mref}~ \group{\Any{REF_0} + \Any{REF_2}}~
  A_{PRM} \\
 G_{\lm, Q_{pob}} & = & g_{Mpob}~ \Any{POB_0}~ A_{PRM}
\end{eqnarray}
 where
\begin{equation}
\elabel{A_pm}
A_{\pm} = \frac{\abs{\Afo{IT_0}}^2 \pm \abs{\Afo{IR_0}}^2}{\Afo{PR_0}},
\end{equation}
%
\begin{equation}
\elabel{A_PRM}
A_{PRM} = \frac{\abs{\Afo{PR_1}}^2}{\Abi{PR_1}},
\end{equation}
%
 and the various $g$s are constant gain coefficients.
The contribution from the second-order sidebands has been
 included only in the reflected signals since,
 while carrier contribution dominates at other ports,
 $\Any{REF_0}$ may go to zero,
 making $\Any{REF_2}$ the dominant contributor to the reflected signal.
Filling in zeros for the (relatively small)
 elements of the sensing matrix not given above
 yields the full sensing matrix,
\begin{equation}
\elabel{SM_PRIFO}
\SM_{PR} =
\matrixBegin{cccc}
 G_{\Lp, I_{ref}} & G_{\Lm, I_{ref}} & G_{\lp, I_{ref}} & 0 \\
 G_{\Lp, I_{pob}} & G_{\Lm, I_{pob}} & G_{\lp, I_{pob}} & 0 \\
 G_{\Lp, Q_{asy}} & G_{\Lm, Q_{asy}} & 0 & 0 \\
 0 & 0 & 0 & G_{\lm, Q_{ref}} \\
 0 & 0 & 0 & G_{\lm, Q_{pob}}
\matrixEnd.
\end{equation}

\newpage
%%%%%%%%%%%%%%%% ND
\tit{Normalized Determinant}

In order to use the determinant of a matrix not just as an indicator
 of arrival at a matrix singularity,
 but as a measure of the proximity of a singularity,
 it must be normailzed.
The normailzed determinant given here is constructed such that
 scaling of any given row or column does not change its value.

For some set of indices $p_{m,n}$,
 the determinant of a matrix $\mathbf{M}$ can be expressed
\begin{equation}
\abs{\mathbf{M}} = \sum_{n = 1}^N (-1)^n
 \prod_{m = 1}^N \mathbf{M}_{m,p_{m, n}}.
\end{equation}
For the same $p_{m,n}$, the normalized determinant is
\begin{equation}
\widehat{\abs{\mathbf{M}}} = \frac
{\sum_{n = 1}^N (-1)^n \prod_{m = 1}^N \mathbf{M}_{m,p_{m, n}}}
{\sum_{n = 1}^N \abs{\prod_{m = 1}^N \mathbf{M}_{m,p_{m, n}}}}
\end{equation}
 such that
\begin{equation}
-1 \le {\widehat{\abs{\mathbf{M}}}} \le 1.
\end{equation}

\end{document}
